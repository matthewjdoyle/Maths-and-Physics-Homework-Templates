% Homework content file

\hrulefill

\section*{Notes/Formulas}

[Additional notes or definitions if needed]

\hrulefill

\section*{Problems}



\begin{problem}{10}
Find the derivative of $f(x) = 3x^4 - 2x^3 + 5x^2 - 7x + 4$. Then, find all values of $x$ where the gradient of the curve is equal to zero. Determine whether each stationary point is a maximum, minimum, or point of inflection.

\begin{solution}
To find the derivative of $f(x) = 3x^4 - 2x^3 + 5x^2 - 7x + 4$, we use the power rule:

$f'(x) = 12x^3 - 6x^2 + 10x - 7$

For stationary points, we set $f'(x) = 0$:
$12x^3 - 6x^2 + 10x - 7 = 0$

This is a cubic equation. Let's factor it:
$12x^3 - 6x^2 + 10x - 7 = 0$

Using numerical methods (or a calculator), the solutions are approximately:
$x \approx -0.719$, $x \approx 0.5$, and $x \approx 0.886$

To determine the nature of each stationary point, we find the second derivative:
$f''(x) = 36x^2 - 12x + 10$

Evaluating at each point:
$f''(-0.719) \approx 28.11 > 0$ → Local minimum
$f''(0.5) \approx 13 > 0$ → Local minimum
$f''(0.886) \approx 20.73 > 0$ → Local minimum

Therefore, all three stationary points are local minima.

\fbox{Stationary points: $x \approx -0.719$, $x \approx 0.5$, and $x \approx 0.886$, all local minima}
\end{solution}
\end{problem}

\begin{problem}{15}
A small metal sphere of mass 0.15 kg is attached to one end of a light inextensible string of length 0.85 m. The other end of the string is fixed to a point on a ceiling. The sphere is released from rest with the string horizontal, and swings in a vertical plane.

\begin{part}{a}{Calculate the speed of the sphere when the string makes an angle of $45^\circ$ with the vertical.}{5}
\begin{solution}
This is a pendulum problem where we can apply conservation of energy.

Initial position: horizontal position ($90^\circ$ from vertical)
Final position: $45^\circ$ from vertical

Let's denote:
- $m = 0.15$ kg (mass of sphere)
- $L = 0.85$ m (length of string)
- $g = 9.8$ m/s²

Initial height from lowest point: $h_i = L(1 - \cos(90^\circ)) = L = 0.85$ m
Final height from lowest point: $h_f = L(1 - \cos(45^\circ)) = L(1 - \frac{\sqrt{2}}{2}) = 0.85(1 - 0.7071) = 0.85 \times 0.2929 = 0.249$ m

Energy conservation:
$mgh_i = mgh_f + \frac{1}{2}mv^2$

Solving for $v$:
$v = \sqrt{2g(h_i - h_f)} = \sqrt{2 \times 9.8 \times (0.85 - 0.249)} = \sqrt{19.6 \times 0.601} = \sqrt{11.78} = 3.43$ m/s

\fbox{$v = 3.43$ m/s}
\end{solution}
\end{part}

\begin{part}{b}{Calculate the tension in the string when it makes an angle of $45^\circ$ with the vertical.}{5}
\begin{solution}
At $45^\circ$ from vertical, we need to account for both centripetal force and the component of weight.

From part (a), the speed at this position is $v = 3.43$ m/s.

The tension $T$ has two components:
1. Counteracting the component of weight along the string: $mg\cos(45^\circ)$
2. Providing centripetal force: $\frac{mv^2}{L}$

Therefore:
$T = mg\cos(45^\circ) + \frac{mv^2}{L}$
$T = 0.15 \times 9.8 \times \frac{\sqrt{2}}{2} + \frac{0.15 \times 3.43^2}{0.85}$
$T = 0.15 \times 9.8 \times 0.7071 + \frac{0.15 \times 11.76}{0.85}$
$T = 1.039 + 2.075 = 3.11$ N

\fbox{$T = 3.11$ N}
\end{solution}
\end{part}

\begin{part}{c}{If the string can withstand a maximum tension of 12 N before breaking, determine the minimum height from which the sphere can be released horizontally without the string breaking at any point in its motion. Assume $g = 9.8~\text{m/s}^2$.}{5}
\begin{solution}
The maximum tension occurs at the bottom of the swing (lowest point). Let's denote the initial height as $h$.

At the lowest point, conservation of energy gives us the speed:
$mgh = \frac{1}{2}mv^2$
$v = \sqrt{2gh}$

The tension at the lowest point is:
$T = mg + \frac{mv^2}{L} = mg + \frac{m(2gh)}{L} = mg + \frac{2mgh}{L}$

Setting $T = 12$ N (maximum):
$12 = 0.15 \times 9.8 + \frac{2 \times 0.15 \times 9.8 \times h}{0.85}$
$12 = 1.47 + \frac{2.94h}{0.85}$
$12 - 1.47 = \frac{2.94h}{0.85}$
$10.53 = \frac{2.94h}{0.85}$
$10.53 \times 0.85 = 2.94h$
$8.95 = 2.94h$
$h = \frac{8.95}{2.94} = 3.04$ m

Since the string length is 0.85 m, the minimum height from which the sphere can be released is:
$3.04 - 0.85 = 2.19$ m above the lowest point of the swing.

\fbox{Minimum height = 2.19 m}
\end{solution}
\end{part}

\end{problem} 